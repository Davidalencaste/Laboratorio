\section{Desarrollo Experimental}

El material utilizado para este experimento fue:
\begin{itemize}
    \item 3 multímetros
    \item 1 cautín de alta potencia
    \item 2 termopares
    \item Alambre de cobre
    \item 1 ventilador
\end{itemize}

El experimento se realizó para 5 longitudes distintas del alambre. Las longitudes fueron:
\begin{itemize}
    \item 25 cm
    \item 20 cm
    \item 15 cm
    \item 10 cm
    \item 5 cm
\end{itemize}

Cada alambre fue lijado en sus extremos para poder medir el voltaje generado por el efecto Seebeck.

\subsection{Procedimiento experimental}

Se conectaron los extremos de cada alambre a las terminales del voltímetro. Asimismo, en cada extremo se colocó un termopar lo más cercano posible al alambre. Una de las puntas (A) se puso en contacto con el cautín; esta sería la punta cuya temperatura aumentaría conforme se calentara el cautín. \\

Una vez montado el arreglo experimental, se encendieron los multímetros, colocando el voltímetro en el rango de mayor precisión posible, que en este caso fue $0.1$ mA. Antes de encender el cautín se comprobó que la temperatura en ambas puntas fuera igual o muy cercana, indicando un buen funcionamiento de los termopares. \\

Después de esta verificación, se encendió el cautín y, utilizando la cámara de un teléfono celular en modo de cámara lenta, se comenzó a grabar la lectura del voltímetro y las temperaturas indicadas por los termopares. El uso de cámara lenta permitió registrar adecuadamente cambios rápidos en las mediciones, que ocurrían en fracciones de segundo. \\

El cautín se mantuvo encendido hasta alcanzar una temperatura máxima de $170^{\circ}$C. Al llegar a este valor, se desconectó el cautín y se encendió el ventilador (sin dejar de registrar datos). Una vez que la punta A alcanzó una temperatura de $40^{\circ}$C, se detuvo la grabación y finalizó la ronda experimental para esa muestra. \\

El ventilador se dejó encendido después de detener la grabación hasta que la punta A regresara a la misma temperatura que la punta B, con el fin de poder proceder con la siguiente muestra.