\section{Conclusión}

En este experimento se obtuvieron varios resultados relevantes, que se resumen a continuación:

\begin{itemize}
    \item \textbf{Dependencia de $S$ respecto a $l$:}  
    No se encontró una relación directa entre estas dos cantidades, como se observa en la Figura \ref{Fig: coef}. Esto sugiere que el movimiento de portadores es esencialmente insensible a la longitud del conductor dentro del rango estudiado. Sin embargo, para confirmar esta afirmación sería necesario obtener más datos empleando una mayor variedad de longitudes.
    
    \item \textbf{Dependencia de $S$ respecto a $T$:}  
    La relación entre el coeficiente de Seebeck y la temperatura fue evidente. El intervalo en el que $S$ se mantuvo aproximadamente constante fue pequeño en comparación con el rango total de temperaturas registradas. Atribuimos este comportamiento a la disminución de la conductividad eléctrica del material conforme aumenta la temperatura, lo cual afecta directamente al movimiento de los portadores de carga.

    \item \textbf{Rangos donde $S$ es constante:}  
    En la segunda familia de gráficas se observa que, para todas las muestras, el voltaje presenta un comportamiento lineal aproximadamente entre $(0 - 45)^\circ$C. Este resultado es especialmente interesante porque sugiere que incluso en termopares reales existe una “desensibilización’’ a temperaturas elevadas, debida a variaciones en sus propiedades termoeléctricas.
\end{itemize}

En general, este trabajo resalta la importancia del efecto Seebeck, un fenómeno físico fundamental cuya aplicación práctica se materializa en los \textit{termopares}, instrumentos esenciales para la medición precisa de temperatura en la física experimental.

Para futuras versiones de este experimento, recomendamos utilizar voltímetros con mayor precisión, ya que con la resolución empleada fue posible obtener únicamente alrededor de 10 datos útiles por muestra. También sería interesante estudiar explícitamente la contribución del término $S_t$ durante cambios rápidos de temperatura y verificar la aparición de picos asociados a variaciones abruptas de la movilidad. Por último, podría explorarse el uso de un método alternativo para medir la temperatura que no involucre termopares, por razones que se omiten para no comprometer la interpretación física del sistema.
