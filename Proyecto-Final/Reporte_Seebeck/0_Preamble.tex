%%%%%%%%%%%%%%%%%%%%%%%%%%%%%%%%%%%%%%%%%%%%%%%%%%%%%%%%%%%%
%%%                                                      %%% 
%%%                Alencaste Desachy David Antonio              %%%
%%%       Universidad Nacional Autónoma de México        %%%
%%%                          2025                        %%%
%%%%%%%%%%%%%%%%%%%%%%%%%%%%%%%%%%%%%%%%%%%%%%%%%%%%%%%%%%%%

%%%%%%%%%%%%%%%%%%%%%%%%%%%%%%%%%%%%%%%%%%%%%%%%%%%%%%%%%%%%
%% DEFINITIONS
%%%%%%%%%%%%%%%%%%%%%%%%%%%%%%%%%%%%%%%%%%%%%%%%%%%%%%%%%%%%

\documentclass[letterpaper]{article}
\usepackage[utf8]{inputenc}
\usepackage[english, spanish, mexico, es-noitemize, es-noindentfirst]{babel}

%%%%%%%%%%%%%%%%%%%%%%%%%%%%%%%%%%%%%%%%%%%%%%%%%%%%%%%%%%%%
%% DOCUMENT FORMAT
%%%%%%%%%%%%%%%%%%%%%%%%%%%%%%%%%%%%%%%%%%%%%%%%%%%%%%%%%%%%
\usepackage{microtype} %improves the spacing between words and letters
\usepackage{hyperref}
\usepackage{imakeidx}         % allows index generation
\usepackage{tgbonum}        %  LaTeX will use Bonum font family
\usepackage{multicol}       % Make Two-Column Text
\usepackage{ragged2e}    
%\usepackage{natbib}% Alinea el texto     % Modifica el comportamiento de \include{}
\usepackage{lipsum}         % Texto de prueba
\urlstyle{same}
\usepackage{microtype}
%%%%%%%%%%%%%%%%%%%%%%%%%%%%%%%%%%%%%%%%%%%%%%%%%%%%%%%%%%%%
%% LANGUAGES
%%%%%%%%%%%%%%%%%%%%%%%%%%%%%%%%%%%%%%%%%%%%%%%%%%%%%%%%%%%%

%\usepackage[english]{babel}
\usepackage[spanish]{babel}

%%%%%%%%%%%%%%%%%%%%%%%%%%%%%%%%%%%%%%%%%%%%%%%%%%%%%%%%%%%%
%% MATH TOOLS
%%%%%%%%%%%%%%%%%%%%%%%%%%%%%%%%%%%%%%%%%%%%%%%%%%%%%%%%%%%%
\usepackage{amsmath, amssymb, amsfonts, amsthm, fouriernc, mathtools}
\usepackage[nolimits]{cmupint}% Integrales rectas
\usepackage{siunitx}% Unidades del SI
%%%%%%%%%%%%%%%%%%%%%%%%%%%%%%%%%%%%%%%%%%%%%%%%%%%%%%%%%%%%
%% COLOR DEFINITIONS
%%%%%%%%%%%%%%%%%%%%%%%%%%%%%%%%%%%%%%%%%%%%%%%%%%%%%%%%%%%%
%\usepackage[svgnames]{xcolor} % Enabling mixing colors and color's call by 'svgnames'
%%%%%%%%%%%%%%%%%%%%%%%%%%%%%%%%%%%%%%%%%%%%%%%%%%%%%%%%%%%%
\newcommand{\textb}{\color{Black} \usefont{OT1}{lmss}{m}{n}}
\newcommand{\black}{\color{black} \usefont{OT1}{lmss}{m}{n}}
\newcommand{\blackb}{\color{black} \usefont{OT1}{lmss}{b}{n}}
\newcommand{\blacks}{\color{black} \usefont{OT1}{qhv}{m}{n}}
\newcommand{\red}{\color{LightCoral} \usefont{OT1}{lmss}{m}{n}}
\newcommand{\green}{\color{Turquoise} \usefont{OT1}{lmss}{m}{n}}


%%%%%%%%%%%%%%%%%%%%%%%%%%%%%%%%%%%%%%%%%%%%%%%%%%%%%%%%%%%%
%%%%  Configuración del documento
%%%%%%%%%%%%%%%%%%%%%%%%%%%%%%%%%%%%%%%%%%%%%%%%%%%%%%%%%%%%
\usepackage[margin=0.75in, headheight=2\baselineskip, headsep=\baselineskip]{geometry}
\usepackage{fancyhdr}
\usepackage[shortlabels, inline]{enumitem}
\usepackage{multicol, array, multirow}
\usepackage{titlesec, apptools}
\usepackage[titles]{tocloft}

\pagestyle{fancy}
\lhead{Laboratorio de Física Contemporánea II}
\rhead{Facultad de Ciencias, UNAM}

%%%%%%%%%%%%%%%%%%%%%%%%%%%%%%%%%%%%%%%%%%%%%%%%%%%%%%%%%%%%
%%%%% Gráficos
\usepackage{graphicx, booktabs, threeparttable, wrapfig, pdfpages, pdflscape}
\usepackage{pgfplots}
\usepackage[auto]{contour}
\usepackage[labelfont=bf, labelsep=period, margin=.05\columnwidth, font=small]{caption}
\usepackage{subcaption}
\graphicspath{{imágenes/}{escudos/}}

%%%%%%%%%%%%%%%%%%%%%%%%%%%%%%%%%%%%%%%%%%%%%%%%%%%%%%%%%%%%%

%%%%%%%%%%%%%%%%%%%%%%%%%%%%%%%%%%%%%%%%%%%%%%%%%%%%%%%%%%%%%
%%%%%%%%%%%%            BIBLIOGRAFIA          %%%%%%%%%%%%%%%
%%%%%%%%%%%%%%%%%%%%%%%%%%%%%%%%%%%%%%%%%%%%%%%%%%%%%%%%%%%%%
\usepackage[backend=biber]{biblatex}
\usepackage{csquotes}
\addbibresource{Bibliografias/Books.bib}
\addbibresource{Bibliografias/Articles.bib}
\addbibresource{Bibliografias/Urls.bib}
%%%%%%%%%%%%%%%%%%%%%%%%%%%%%%%%%%%%%%%%%%%%%%%%%%%%%%%%%%%%%%%%%%%%%
% Versión de PGFplots
\pgfplotsset{compat=newest}



% Todos los enlaces del documento en negro
\hypersetup{allcolors=black}

% Listas personalizadas
\setlist[enumerate, 1]{%
	label={\textbf{\color{emphcolor}\arabic*.}},
	ref=\textbf{\arabic*}
}
\setlist[enumerate, 2]{%
	label={\textbf{\color{emphcolor}(\alph*)}},
	ref=\textbf{(\alph*)}
}
\setlist[enumerate, 3]{%
	label={\color{emphcolor}\roman*.},
	ref=\roman*
}

\setlist[itemize, 1]{%
	label={\color{emphcolor}\textbullet},
}
\setlist[itemize, 2]{%
	label={\color{emphcolor}--},
}
\setlist[itemize, 3]{%
	label={\color{emphcolor}*},
}

% Redefinición de "y otros" en la bibliografía
\DefineBibliographyStrings{spanish}{andothers={et~al\adddot}}

% Notas de tablas en tamaño footnote
\appto\TPTnoteSettings{\footnotesize}

%%%%%%%%%%%%%%%%%%%%%%%%%%%%%%%%%%%%%%%%%%%%%%%%%%%%%%%%%%%%%%5
%%%%%%%%%%%%  COMANDOS MATEMATICOS ADICIONALES %%%%%%%%%%%%%%%
\newcommand{\bfvec}[1]{\vec{\textbf{#1}}}
\newcommand{\pqty}[1]{\left( #1 \right)}
\newcommand{\bqty}[1]{\left[ #1 \right]}
\newcommand{\Bqty}[1]{\left\lbrace #1 \right\rbrace}
\newcommand{\abs}[1]{\left\lvert #1 \right\rvert}
\newcommand{\norm}[1]{\left\lvert\left\lvert #1 \right\rvert\right\rvert}
\newcommand{\rtwovec}[2]{% Vectores columna en R^2 
	\begin{bmatrix*}[r]#1 \\ #2 
	\end{bmatrix*}}
\newcommand{\rthreevec}[3]{% Vectores columna en R^3
	\begin{bmatrix*}[r]
		#1 \\ #2 \\ #3
	\end{bmatrix*}}
\newcommand{\bvec}[1]{% Vectores
	\mathbf{#1}}
\newcommand{\uvec}[1]{% Vectores unitarios (con gorrito)
	\textbf{\^{#1}}}
\newenvironment{amatrix}[2]{% Matices aumentadas
	\left[\begin{array}{@{}*{#1}{r}|*{#2}{r}@{}}
	}%
	{
	\end{array}\right]}
\DeclareMathOperator{\mcd}{mcd}% Máximo común divisor
\DeclareMathOperator{\midd}{med}% Número medio

% Descomentar esta línea si se elige fouriernc para la fuente matemática y emplear en vez de \nabla
%\newcommand{\Nabla}{\scalebox{1.075}{$\nabla$}}

%%%%%%%%%%%%%%%%%%%%%%%%%%%%%%%%%%%%%%%%%%%%%%%%%%%%%%%%%%%%555
%%%%%%%%%%%%%%%%%%%%%%%%%%%%%%%%%%%%%%%%%%%%%%%%%%%%%%%%%%%%%%%

% Colores
\definecolor{emphcolor}{HTML}{003D64}% Color de énfasis
\definecolor{auxcolor}{HTML}{E8442A}% Color auxiliar

% Diseño de títulos de sección
\titleformat{\section}
{\normalfont\Large\bfseries}{\thesection.}{1em}{\color{black}}

\titleformat{\subsection}
{\normalfont\large\bfseries}{\thesubsection.}{1em}{\color{black}}

\titleformat{\subsubsection}
{\normalsize\bfseries}{\thetitle.}{\widthof{\enspace}}{\normalsize\color{emphcolor}}

\AtAppendix{%
    \titleformat{\section}[block]
    {\normalfont\Large\bfseries}{\appendixname\ \thesection.}{1em}{\color{emphcolor}}
}

% Tipos de columna útiles
\newcolumntype{P}[1]{>{\centering\arraybackslash}p{#1}}
\newcolumntype{M}[1]{>{\centering\arraybackslash}m{#1}}
\newcolumntype{R}[1]{>{\arraybackslash}m{#1}}

% Rediseño del estilo plain
\fancypagestyle{plain}{%
	\renewcommand{\headrulewidth}{0pt}
	
	\fancyhf{}
	\fancyfoot[c]{%
    	\begin{tabular}{M{1em}@{}M{2em}@{}M{1em}}
    		\textcolor{emphcolor}{$\boldsymbol{-}$}
    		& \thepage &
    		\textcolor{emphcolor}{$\boldsymbol{-}$}
    	\end{tabular}
    }
}
%%%%%%%%%%%%%%%%%%%%%%%%%%%%%%%%%%%%%%%%%%%%%%%%%%%%%%%%%%%%%%%%%%
%%%%%%%%%%%%%%%%%%%%%%%%%%%%%%%%%%%%%%%%%%%%%%%%%%%%%%%%%%%%%%%%%%
%%%%%%%%%                   PORTADA                      %%%%%%%%%   '
%%%%%%%%%%%%%%%%%%%%%%%%%%%%%%%%%%%%%%%%%%%%%%%%%%%%%%%%%%%%%%%%%%
%%%%%%%%%%%%%%%%%%%%%%%%%%%%%%%%%%%%%%%%%%%%%%%%%%%%%%%%%%%%%%%%%% 
% Diseño de los encabezados y pies de página
\fancypagestyle{fancy}{%
    \xpretocmd\headrule{\color{emphcolor}}{}{\PatchFailed}% Cambio de color de la regla en el encabezado
    
    \fancyhf{}
    \fancyhead[le]{%
    	\begin{tabular}{@{} R{4ex} @{\ } R{\widthof{\small\shortinstitute\ }}}
    		\includegraphics[width=4ex]{fcunam}% Escudo de la instittución
    		& \small\shortinstitute
    	\end{tabular}
    }
    \fancyhead[re]{%
    	\small\textit{\subject}% Nombre de la asignatura
    }
    
    \fancyhead[co]{%
        \begin{tabular}{@{} R{4ex} R{\widthof{\small\subtitle\ }} R{4ex} @{}}
            \phantom{\includegraphics[width=4ex]{fcunam}} & \small\textit{\subtitle} & \phantom{\includegraphics[width=4ex]{fcunam}}
        \end{tabular}
    }
    
    \fancyfoot[c]{%
    	\begin{tabular}{M{1em} @{} M{2em} @{} M{1em}}
    		\textcolor{emphcolor}{$\boldsymbol{-}$}
    		& \thepage &
    		\textcolor{emphcolor}{$\boldsymbol{-}$}
    	\end{tabular}
    }
}

% Comandos para ingresar keywords en el abstract
\newcommand{\espkeywords}{\noindent\textit{Palabras clave:\ }}
\newcommand{\engkeywords}{\noindent\textit{Keywords:\ }}

%%%%%%%%%%%%%%%%%%%%%%%%%%%%%%%%%%%%%%%%%%%%
%%%%%%%%%%% DATOS DE LA PORTADA %%%%%%%%%%%%
%%%%%%%%%%%%%%%%%%%%%%%%%%%%%%%%%%%%%%%%%%%%

\author{Alencaste Desachy David Antonio}

\title{Proyecto Final}%%%%%%% PRACTICA %%%%%%%%%%
\newcommand{\subtitle}%%%%%%% TITULO %%%%%%%%%%
	{Efecto Seebeck para distintas longitudes de un alambre conductor}
    
\date{\today}

\newcommand{\subject}%%%%%%% MATERIA %%%%%%%%%%
	{Laboratorio de Física Contemporánea II}

\newcommand{\maininstitute}%%%%%%%% INSTITUTO %%%%%%%%%%
	{Facultad de Ciencias - Física }

\newcommand{\secinstitute}%%%%%%%% UNIVERSIDAD %%%%%%%%%%
	{Universidad Nacional Autónoma de México}
	
\makeatletter 
	\renewcommand\maketitle{%
		\thispagestyle{plain}
		\begin{center}
    		\begin{tabular}{@{} M{.175\textwidth} @{} M{.65\textwidth} @{} M{.175\textwidth} @{}}
    		    \includegraphics[width=.14\textwidth]{unam} & \Large\textcolor{emphcolor}{\textbf{\@title}} \par \huge\textbf{\subtitle} & \includegraphics[width=.14\textwidth]{fcunam}
    		\end{tabular}
			
			\bigskip
			
			\normalsize
            \begin{center}
			\begin{tabular}{@{}c@{\qquad}c@{\qquad}c@{}}
				David Antonio Alencaste Desachy\\
				{\footnotesize\href{mailto:dalencaste@ciencias.unam.mx}{\color{emphcolor}\texttt{dalencaste@ciencias.unam.mx}}} 
                
			\end{tabular}
            \end{center}
            
                Profesor Asesor:\\ Dra. Beatriz Elizabeth Fuentes Madariaga\\
			
			\bigskip
			\subject\\
			\maininstitute\\
			\secinstitute
			
			\medskip
			
			\textit{\today}
		\end{center}
	}
 \makeatother


%%%%%%%%%%%%%%%%%%%%%%%%%%%%%%%%%%%%%%%%%%%%%%%%%%%%%%%%%%%%%%%%%%
%%%%%%%%%%%%%%%%%%%%%%%%%%%%%%%%%%%%%%%%%%%%%%%%%%%%%%%%%%%%%%%%%%
%%%%%%%%%%                  DOCUMENTO                   %%%%%%%%%%
%%%%%%%%%%%%%%%%%%%%%%%%%%%%%%%%%%%%%%%%%%%%%%%%%%%%%%%%%%%%%%%%%%
%%%%%%%%%%%%%%%%%%%%%%%%%%%%%%%%%%%%%%%%%%%%%%%%%%%%%%%%%%%%%%%%%%

\begin{document} 
    \maketitle
    %\vspace*{\fill}

\begin{center}
	\color{emphcolor}\rule{\textwidth}{0.4pt}%
\end{center}
\vspace{-14pt}
\begin{abstract}
% Resumen
    En este trabajo se estudió el efecto Seebeck en alambres de cobre de distintas longitudes, midiendo la relación entre la diferencia de temperatura y el voltaje inducido. Se registraron datos durante el calentamiento y enfriamiento del conductor, y se analizaron únicamente los intervalos donde el voltaje mostró un comportamiento creciente. Los resultados indican que el coeficiente de Seebeck se mantiene aproximadamente constante solo en un rango reducido de temperaturas, mientras que a temperaturas mayores su variación se asocia a cambios en la conductividad del material. No se observó una dependencia clara de $S$ con la longitud del alambre. El estudio resalta la relevancia experimental del efecto Seebeck y su aplicación en dispositivos termoeléctricos como los termopares.


    \medskip

    
    \espkeywords Efecto Seebeck, termopares, gradiente de temperatura, coeficiente de Seebeck, conducción eléctrica, materiales conductores, voltaje termoeléctrico.
	\end{abstract}
%	\selectlanguage{english}
%	\begin{abstract}
	%abstract
     
%		\medskip
		
%		\engkeywords 
%	\end{abstract}
%	\vspace{-14pt}
	\begin{center}
		\color{emphcolor}\rule{\textwidth}{0.4pt}%
	\end{center}
%	\selectlanguage{spanish}
%\vspace*{\fill}
		

	 
	
    \begin{multicols*}{2}
    \section{Introducción}

En 1821, Thomas Johann Seebeck descubrió un fenómeno nuevo en los conductores al ser expuestos a un gradiente de temperatura.\footnote{Aunque a veces se menciona a Volta en la historia de la termoeléctrica, el descubrimiento formal del efecto corresponde a Seebeck.} \\ 
Encontró que, a partir de una diferencia de temperatura en un conductor, se genera una diferencia de potencial eléctrico, cuantificada mediante el \textit{coeficiente de Seebeck}. \\

El efecto Seebeck consiste en la generación de un movimiento neto de portadores de carga (en conductores y semiconductores) debido a un gradiente de temperatura, permitiendo la conversión de energía térmica en energía eléctrica. \\ 
Bajo la acción de dicho gradiente, los huecos en un material tipo p se difunden hacia el material tipo n, mientras que los electrones del material tipo n se difunden hacia el material tipo p, generando así una fuerza electromotriz. \cite{Adresi2023}\\

El campo eléctrico asociado a esta fuerza viene dado por la Ecuación~\ref{eq: campo-electrico}:
\begin{equation}
    E = S \cdot \nabla T
    \label{eq: campo-electrico}
\end{equation}
donde $S$ es el \textit{coeficiente de Seebeck}. \\

A partir de esta relación se obtiene:
\[
\Delta \phi = -\int E \cdot dl = \int_{T_1}^{T_2} (S_A(T) - S_B(T))\, dT
\]

\begin{equation}
    V = \int_{T_1}^{T_2} (S_A(T) - S_B(T))\, dT
\end{equation}

\textit{Suponiendo que $S$ es constante:}
\begin{equation}
    \frac{V}{\Delta T} = (S_A - S_B)
    \label{eq: coef-constante}
\end{equation}


\begin{minipage}{\columnwidth}
    \centering
    \captionsetup{type=figure}
    \includegraphics[width=1.\columnwidth]{Imágenes/diagrama_zeebeck.png}
    \caption{Diagrama del efecto Zeebeck. Del lado izquierdo, de mayor temperatura hay portadores con energía mayor que se escapan al lado de menor temperatura.\\Figura obtenida de \cite{Adresi2023}}
    \label{Fig: EsquemaPlacas}
\end{minipage}

\vspace{.5cm}

En general, sin embargo, se tiene que $S = S(T)$, es decir, depende de la temperatura. \\

En el artículo \cite{Sun2015LargeSeebeck} se demuestra que un gradiente fuerte en la movilidad de los portadores de carga con la temperatura genera un aporte adicional al efecto Seebeck, el cual puede ser dominante frente a otras contribuciones. Como resultado, el coeficiente total de Seebeck puede escribirse como:

\begin{equation}
    S = S_N + S_t
\end{equation}

donde $S_N$ es la contribución convencional y $S_t$ es la contribución asociada al cambio de movilidad con la temperatura. 
    \section{Objetivos}

\subsection{Objetivo General}
Determinar la dependencia del coeficiente de Seebeck ($S$) con la longitud de un alambre conductor.

\subsubsection{Objetivos Particulares}
\begin{enumerate}
    \item Obtener la gráfica de la relación $S(\Delta T, l)$, donde $\Delta T$ es el cambio de temperatura absoluta entre los extremos del alambre conductor y $l$ es su longitud.
    \item Verificar en qué rangos de $\Delta T$ y valores de $l$ el coeficiente de Seebeck se aproxima más a un valor constante, como predice la teoría.
    \item Determinar las posibles causas por las que $S$ deja de comportarse como una constante.
\end{enumerate}
    \input{2.2_Hipotesis}
    \section{Desarrollo Experimental}

El material utilizado para este experimento fue:
\begin{itemize}
    \item 3 multímetros
    \item 1 cautín de alta potencia
    \item 2 termopares
    \item Alambre de cobre
    \item 1 ventilador
\end{itemize}

El experimento se realizó para 5 longitudes distintas del alambre. Las longitudes fueron:
\begin{itemize}
    \item 25 cm
    \item 20 cm
    \item 15 cm
    \item 10 cm
    \item 5 cm
\end{itemize}

Cada alambre fue lijado en sus extremos para poder medir el voltaje generado por el efecto Seebeck.

\subsection{Procedimiento experimental}

Se conectaron los extremos de cada alambre a las terminales del voltímetro. Asimismo, en cada extremo se colocó un termopar lo más cercano posible al alambre. Una de las puntas (A) se puso en contacto con el cautín; esta sería la punta cuya temperatura aumentaría conforme se calentara el cautín. \\

Una vez montado el arreglo experimental, se encendieron los multímetros, colocando el voltímetro en el rango de mayor precisión posible, que en este caso fue $0.1$ mA. Antes de encender el cautín se comprobó que la temperatura en ambas puntas fuera igual o muy cercana, indicando un buen funcionamiento de los termopares. \\

Después de esta verificación, se encendió el cautín y, utilizando la cámara de un teléfono celular en modo de cámara lenta, se comenzó a grabar la lectura del voltímetro y las temperaturas indicadas por los termopares. El uso de cámara lenta permitió registrar adecuadamente cambios rápidos en las mediciones, que ocurrían en fracciones de segundo. \\

El cautín se mantuvo encendido hasta alcanzar una temperatura máxima de $170^{\circ}$C. Al llegar a este valor, se desconectó el cautín y se encendió el ventilador (sin dejar de registrar datos). Una vez que la punta A alcanzó una temperatura de $40^{\circ}$C, se detuvo la grabación y finalizó la ronda experimental para esa muestra. \\

El ventilador se dejó encendido después de detener la grabación hasta que la punta A regresara a la misma temperatura que la punta B, con el fin de poder proceder con la siguiente muestra.
    \section{Resultados y Discusión}

Se presentan dos subconjuntos de los datos obtenidos, ya que cada uno aporta información relevante que conviene analizar por separado.

En primer lugar, se muestran todos los valores registrados durante el experimento, es decir, tanto en la etapa donde la temperatura aumentó como en aquella donde disminuyó mediante el ventilador. En esta parte no se tomaron mediciones de tiempo, de modo que en las siguientes gráficas el eje de las abscisas representa únicamente el número de dato registrado. El objetivo principal aquí es observar el comportamiento del voltaje conforme la temperatura se modifica con el tiempo.

En todas las gráficas aparece una línea roja vertical punteada, la cual indica el punto a partir del cual la temperatura dejó de incrementarse y comenzó a descender debido al ventilador.

\begin{center}
\begin{minipage}{0.95\columnwidth}
    \centering
    \captionsetup{type=figure}
    \includegraphics[width=.9\linewidth]{Imágenes/all_10.png}
    \caption{Datos experimentales $V$ vs. número de dato para un alambre de Cobre de 10\,cm. Se observa un voltaje atípicamente alto en el dato 63.}
\end{minipage}
\end{center}

\begin{center}
\begin{minipage}{0.95\columnwidth}
    \centering
    \captionsetup{type=figure}
    \includegraphics[width=.9\linewidth]{Imágenes/all_15.png}
    \caption{Datos experimentales $V$ vs. número de dato para un alambre de Cobre de 15\,cm. Se aprecia un aumento de voltaje inmediatamente después de iniciar el enfriamiento.}
\end{minipage}
\end{center}

\begin{center}
\begin{minipage}{0.95\columnwidth}
    \centering
    \captionsetup{type=figure}
    \includegraphics[width=.9\linewidth]{Imágenes/al_20.png}
    \caption{Datos experimentales $V$ vs. número de dato para un alambre de Cobre de 20\,cm. Se observa un incremento de voltaje previo al enfriamiento, así como un comportamiento aproximadamente lineal similar al de los datos iniciales.}
\end{minipage}
\end{center}

\begin{center}
\begin{minipage}{0.95\columnwidth}
    \centering
    \captionsetup{type=figure}
    \includegraphics[width=.9\linewidth]{Imágenes/al_25.png}
    \caption{Datos experimentales $V$ vs. número de dato para un alambre de Cobre de 25\,cm. También se observa un aumento de voltaje antes del inicio del enfriamiento.}
\end{minipage}
\end{center}

En todas las muestras se presenta un comportamiento común: conforme la temperatura aumenta, el voltaje alcanza un máximo, pero al seguir incrementándose la temperatura el voltaje ya no aumenta como se esperaría para un coeficiente $S$ constante (Ecuación \ref{eq: coef-constante}). En lugar de ello, disminuye hasta cierto punto y posteriormente vuelve a incrementarse, especialmente cuando la temperatura empieza a descender.

Atribuimos este comportamiento a una característica propia de los conductores: su conductividad disminuye conforme aumenta la temperatura. Esto provoca que los portadores de carga experimenten una mayor resistencia al movimiento, lo que reduce su eficiencia de transporte y, en consecuencia, disminuye el voltaje inducido. 

Cuando la temperatura comienza a descender, la conductividad aumenta nuevamente, facilitando el movimiento de los portadores y elevando el número efectivo de estos en cada extremo del conductor. Esto explica el incremento de voltaje observado durante la etapa de enfriamiento.


Los datos correspondientes a temperaturas más bajas no se registraron por falta de tiempo; sin embargo, cualitativamente en todas las muestras se observó primero un aumento y luego una disminución del voltaje hasta llegar a $0\,\text{V}$.

A continuación se muestran las gráficas de $V$ vs. $T$ utilizando únicamente la región donde el voltaje presenta un comportamiento creciente. Los datos posteriores no se incluyen aquí, ya que el propósito de esta sección es analizar el coeficiente de Seebeck $S$ como función de la temperatura y no del tiempo.

\begin{center}
\begin{minipage}{0.95\columnwidth}
    \centering
    \captionsetup{type=figure}
    \includegraphics[width=.9\linewidth]{Imágenes/regres_10.png}
    \caption{Gráfica experimental $V$ vs. $T$ para un alambre de Cobre de 10\,cm. El ajuste lineal antes de $40^\circ\text{C}$ presenta un coeficiente $R^2=0.95$.}
\end{minipage}
\end{center}

\begin{center}
\begin{minipage}{0.95\columnwidth}
    \centering
    \captionsetup{type=figure}
    \includegraphics[width=.9\linewidth]{Imágenes/regres_15.png}
    \caption{Gráfica experimental $V$ vs. $T$ para un alambre de Cobre de 15\,cm. El ajuste lineal antes de $40^\circ\text{C}$ arroja un coeficiente $R^2=0.89$.}
\end{minipage}
\end{center}

\begin{center}
\begin{minipage}{0.95\columnwidth}
    \centering
    \captionsetup{type=figure}
    \includegraphics[width=.9\linewidth]{Imágenes/regres_20.png}
    \caption{Gráfica experimental $V$ vs. $T$ para un alambre de Cobre de 20\,cm. El ajuste lineal antes de $40^\circ\text{C}$ presenta un coeficiente $R^2=0.92$.}
\end{minipage}
\end{center}

\begin{center}
\begin{minipage}{0.95\columnwidth}
    \centering
    \captionsetup{type=figure}
    \includegraphics[width=.9\linewidth]{Imágenes/regres_25.png}
    \caption{Gráfica experimental $V$ vs. $T$ para un alambre de Cobre de 25\,cm. El ajuste lineal antes de $65^\circ\text{C}$ arroja un coeficiente $R^2=0.91$.}
\end{minipage}
\end{center}

Las temperaturas utilizadas para realizar las regresiones lineales se eligieron de modo que el valor de $R^2$ se maximizara, pues justamente en ese intervalo la relación entre $V$ y $T$ se aproxima más a la linealidad, es decir, es la región donde el coeficiente de Seebeck puede considerarse aproximadamente constante.

Finalmente, en la Figura \ref{Fig: coef} se muestran los valores obtenidos del coeficiente de Seebeck para cada muestra. No se aprecia una relación clara entre los valores obtenidos y la longitud del alambre, por lo que sería necesario realizar un mayor número de mediciones para obtener un análisis más robusto.

\begin{center}
\begin{minipage}{0.95\columnwidth}
    \centering
    \captionsetup{type=figure}
    \includegraphics[width=.9\linewidth]{Imágenes/coef_S.png}
    \caption{Coeficientes de Seebeck obtenidos para las distintas longitudes del alambre de Cobre. No se identifica una tendencia evidente.}
    \label{Fig: coef}
\end{minipage}
\end{center}

    \section{Conclusión}

En este experimento se obtuvieron varios resultados relevantes, que se resumen a continuación:

\begin{itemize}
    \item \textbf{Dependencia de $S$ respecto a $l$:}  
    No se encontró una relación directa entre estas dos cantidades, como se observa en la Figura \ref{Fig: coef}. Esto sugiere que el movimiento de portadores es esencialmente insensible a la longitud del conductor dentro del rango estudiado. Sin embargo, para confirmar esta afirmación sería necesario obtener más datos empleando una mayor variedad de longitudes.
    
    \item \textbf{Dependencia de $S$ respecto a $T$:}  
    La relación entre el coeficiente de Seebeck y la temperatura fue evidente. El intervalo en el que $S$ se mantuvo aproximadamente constante fue pequeño en comparación con el rango total de temperaturas registradas. Atribuimos este comportamiento a la disminución de la conductividad eléctrica del material conforme aumenta la temperatura, lo cual afecta directamente al movimiento de los portadores de carga.

    \item \textbf{Rangos donde $S$ es constante:}  
    En la segunda familia de gráficas se observa que, para todas las muestras, el voltaje presenta un comportamiento lineal aproximadamente entre $(0 - 45)^\circ$C. Este resultado es especialmente interesante porque sugiere que incluso en termopares reales existe una “desensibilización’’ a temperaturas elevadas, debida a variaciones en sus propiedades termoeléctricas.
\end{itemize}

En general, este trabajo resalta la importancia del efecto Seebeck, un fenómeno físico fundamental cuya aplicación práctica se materializa en los \textit{termopares}, instrumentos esenciales para la medición precisa de temperatura en la física experimental.

Para futuras versiones de este experimento, recomendamos utilizar voltímetros con mayor precisión, ya que con la resolución empleada fue posible obtener únicamente alrededor de 10 datos útiles por muestra. También sería interesante estudiar explícitamente la contribución del término $S_t$ durante cambios rápidos de temperatura y verificar la aparición de picos asociados a variaciones abruptas de la movilidad. Por último, podría explorarse el uso de un método alternativo para medir la temperatura que no involucre termopares, por razones que se omiten para no comprometer la interpretación física del sistema.

    \section*{Agradecimientos}

Fue un gusto y una alegría escribir este reporte, mi último reporte de la carrera en Física. 
Siempre atesoraré los días de alegría y estrés que daban las sesiones de laboratorio, y 
posteriormente la terrible decepción en casa al descubrir que los datos medidos eran 
horribles y no quedaba más que reportar incertidumbres del orden del $300\%$.  

Gracias, Física experimental; eres hermosa a tu manera.

    \input{bibliography}
    \input{Apendices}
    \nocite{*}
    \end{multicols*}
    \clearpage
    %\input{Apendices}
   
\end{document}