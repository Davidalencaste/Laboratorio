%\vspace*{\fill}

\begin{center}
	\color{emphcolor}\rule{\textwidth}{0.4pt}%
\end{center}
\vspace{-14pt}
\begin{abstract}
% Resumen
    En este trabajo se estudió el efecto Seebeck en alambres de cobre de distintas longitudes, midiendo la relación entre la diferencia de temperatura y el voltaje inducido. Se registraron datos durante el calentamiento y enfriamiento del conductor, y se analizaron únicamente los intervalos donde el voltaje mostró un comportamiento creciente. Los resultados indican que el coeficiente de Seebeck se mantiene aproximadamente constante solo en un rango reducido de temperaturas, mientras que a temperaturas mayores su variación se asocia a cambios en la conductividad del material. No se observó una dependencia clara de $S$ con la longitud del alambre. El estudio resalta la relevancia experimental del efecto Seebeck y su aplicación en dispositivos termoeléctricos como los termopares.


    \medskip

    
    \espkeywords Efecto Seebeck, termopares, gradiente de temperatura, coeficiente de Seebeck, conducción eléctrica, materiales conductores, voltaje termoeléctrico.
	\end{abstract}
%	\selectlanguage{english}
%	\begin{abstract}
	%abstract
     
%		\medskip
		
%		\engkeywords 
%	\end{abstract}
%	\vspace{-14pt}
	\begin{center}
		\color{emphcolor}\rule{\textwidth}{0.4pt}%
	\end{center}
%	\selectlanguage{spanish}
%\vspace*{\fill}
		

	 
	